\documentclass[11pt]{article}

\usepackage[a4paper, margin=1in]{geometry}

% FONTS
\usepackage[T1]{fontenc}
\usepackage{charter}  % Font face

% LAYOUT & SPACING
\usepackage{titlesec}
\usepackage{setspace}
\usepackage{fancyhdr}
\usepackage{parskip}  % no paragraph indent + space between paragraphs
\renewcommand{\baselinestretch}{1.25}  % line spacing
\setcounter{secnumdepth}{4}  % max. heading depth


% MISC Packages
\usepackage{booktabs}  % table rules
\usepackage{tabularx}
\usepackage{hyperref}  % Links in TOC and \refs
\usepackage{apacite}

% HEADER & FOOTER
\pagestyle{fancy}
\fancyhf{}
\rhead{pyFin Sentiment: Financial Sentiment Analysis}
\lhead{Moritz Wilksch}
\cfoot{\thepage}

% MISC COMMANDS
\hypersetup{linktocpage, pdfborder = {0 0 0}} % Links
\hypersetup{bookmarks=true, bookmarksnumbered=true} % Links in TOC


% =============================================================================
\begin{document}

\tableofcontents

\newpage


\section{Introduction}
We wish to give your master or bachelor thesis a consistent, high-quality appearance. We therefore ask that authors/students follow some basic guidelines. In essence, you should format your paper exactly like this document. The easiest way to use this template is to download it from the chair webpage and replace the content with your own material. The template file contains specially formatted styles (e.g., Normal, Heading, Bullet, Table Text, References, Title, Author) that will reduce the work in formatting your text. 
Also transformer models (\cite{attentionIsAllYouNeed}) are awesome.

\section{Formatting}
\subsection{Page Size}
On each page, your material (not including the header and footer) should fit within a rectangle of 21 x 29.7 cm (8.267 x 11.692 in.), centered on a A4 paper, beginning 1.9 cm (.75 in.) from the top of the page. Please adhere to the A4 paper size only (hopefully Word or other word processors can help you with it). Right margins should be justified, not ragged. Beware, especially when using this template on a Macintosh, Word may change these dimensions in unexpected ways.

\subsection{Typeset Text}
Prepare your submissions on a word processor or typesetter. Please note that page layout may change slightly depending upon the printer you have specified.

\subsection{Normal or Body Text}
Please use a 12-point Times font or, if it is unavailable, another proportional font with serifs, as close as possible in appearance to Times 12-point with a 1,15-line spacing. Please use sans-serif or non-proportional fonts only for special purposes, such as headings or source code text.
\subsection{Title}
Your paper’s title should be in Times 18-point bold. Ensure proper capitalization within your title (i.e. “The Next Frontier of Information Systems” versus “the next frontier of Information systems”).

\section{Sections}
The heading of a section should be in Times 12-point bold, all in capitals (Heading 1) Style in this template file. Sections should be numbered!

% ---------------
\section{Theoretical Background}
The widely dissemination of social networking sites (SNSs) over the past decade has changed the way of modern communication. Owning mobile devices, SNS use is independent from time and location and thus has become a highly integrated part of most people’s daily routine. […]

\subsection{Subsection}
Headings of subsections should be in Times 12-point bold with initial letters capitalized (Heading 2). (Note: for sub-sections and sub-subsections, a word like ‘the’ or ‘of’ is not capitalized unless it is the first word of the heading.) Subsections should be numbered with respect to section’s number!
\subsubsection{Sub-subsections}
Headings for sub-subsections should be in Times 10-point italic with initial letters capitalized (Heading 3). Please do not go any further into another layer/level.

\subsection{Social Networking Sites (Subsection Example)}
Social Networking Sites (SNSs) are web-based services that allow members to create profiles and connect with other members (Ellison et al., 2007:1143). By participating in SNSs users can build and maintain their online personal network. SNSs enables users to quickly exchange messages, pictures, contact information or other content in synchronous or asynchronous forms of communication. […]

\subsubsection{Literature Search Strategy (Subsection Example)}
The search topic was defined as the impact of SNS usage on user’s SWB. The search structure consisted of two concept groups: SNS (1) and SWB (2). […]

\section{Methodology}
Place figures and tables close to the relevant text (or where they are referenced in the text).
Captions should be Times 11-point bold (Caption Style in this template file). They should be numbered (e.g., ``Table 1'' or “Figure 2”), centered and placed beneath the figure or table. Please note that the words “Figure” and “Table” should be spelled out (e.g., “Figure” rather than “Fig.”) wherever they occur.
Color figures are possible.
\subsection{Inserting Images}
Occasionally MS Word generates larger-than-necessary PDF files when images inserted into the document are manipulated in MS Word. To minimize this problem, use an image editing tool to resize the image at the appropriate printing resolution (usually 300 dpi), and then insert the image into Word using Insert | Picture | From File...
\newline
Table \ref{table:treatment} shows some information. Take a look:
\begin{table}[!h]
	\centering
	\begin{tabular}{lcc}
		\hline
		             & \textbf{Treatment A} & \textbf{Treatment B} \\
		\hline
		John Smith   & 1           & 2           \\
		Jane Doe     & --          & 3           \\
		Mary Johnson & 4           & 5           \\
		\hline
	\end{tabular}
	\caption{A table without vertical lines.}
	\label{table:treatment}
\end{table}

The reviewing process will be double blind at the level of reviewers and area chairs (i.e., reviewers and area chairs cannot see author identities) but not at the level of senior area chairs and \emph{program chairs}. As an \textbf{author}, you are responsible for anonymizing your submission. In particular, you should not include author names, author affiliations, or acknowledgements in your submission and you should avoid providing any other identifying information (even in the supplementary material). If you need to cite one of your own papers, you should do so with adequate anonymization to preserve double-blind reviewing (e.g., write “In the previous work of Author et al. [1]…” rather than “In our previous work [1]...”). If you need to cite one of your own papers that is in submission to NeurIPS or elsewhere please do so with adequate anonymization and make sure the cited submission is available for reviewers to read (e.g., if the cited submission is available as a non-anonymous preprint, then write “Author et al. [1] concurrently show…”; if the cited submission is not available as a non-anonymous preprint, then include a copy of the cited submission in the supplementary material and write “Anonymous et al. [1] concurrently show...”).

Also, it has been shown that balbalsdjf and \citeA{attentionIsAllYouNeed} are wrong \cite{chua1993cnn}.

\newpage
\bibliography{src/bibliography/refs}{}
\bibliographystyle{apacite}

\end{document}