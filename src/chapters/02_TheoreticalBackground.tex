\section{Theoretical Background}



% =======================================================================
\subsection{Sentiment Analysis}  % TODO: wording
%todo: text here?
% -----------------------------------------------------------------------
\subsubsection{Operationalization of Sentiment}
% todo: vvv weak line of argument
Before surveying the literature on automated sentiment analysis models, the term ``Sentiment'' and its operationalization in the scientific literature need to be clearly defined. \shortciteA{munezero2014they} note that the terms \emph{emotion}, \emph{sentiment}, and \emph{opinion} are often used interchangeably, especially in the literature on NLP. However, there are important distinctions to be made between these terms. Emotions can be seen as moods that are not very long-lasting and are caused ``when we perceive positive or negative significant changes in our personal situation'' \shortcite[p.~3]{ben2001subtlety}. On the other hand, a sentiment is defined as ``an acquired and relatively permanent major neuropsychic disposition'' \cite[p.~16]{cattell1940sentiment}. This renders sentiments different from opinions, which are ``personal interpretations of information formed in the mind'' \cite[p.~4]{munezero2014they} and thus require a specific piece of information to interpret. %todo: wording
Despite the differences in the definition of these terms, they are used in ambiguous ways in the field of NLP. In particular, sentiment analysis and opinion mining refer to the same area of research \cite{liu2012book} and are more prevalent than studies on emotion extraction \shortcite{ravi2015survey}, \textcolor{red}{[ too long]} potentially because emotions are a more complex construct and cannot be fully conveyed through text \shortcite{munezero2014they}.

Another challenge is the operationalization of emotions, sentiments, or opinions. For example, emotion extraction studies like the ones conducted by \shortciteA{li2014text} or \shortciteA{aman2007identifying} tend to frame the problem as a six-class classification task where each class corresponds to one of the six basic emotions proposed by \shortciteA{ekman1971constants}. These six emotions are happiness, sadness, anger, surprise, disgust, and fear. However, \shortciteA{aman2007identifying} point out that the inter-rater reliability for this classification task on a corpus of text can be as low as 60\%, indicating that emotion extraction is a non-trivial task even for human annotators. There are variations of this measurement scale, like the ``Profile of Mood States'' questionnaire \shortcite{mcnair1971manual} which measures a different set of six emotional states. What all of these scales have in common is that they are multi-dimensional measurements to assess emotional states. \newline
On the other hand, the literature on sentiment analysis and opinion mining often employs simpler, even one-dimensional scales. The most prominent way of operationalizing sentiment is sentiment \emph{polarity}, which categorizes each unit of analysis as positive, negative, or, in some cases, neutral \cite{ravi2015survey}. Other approaches try to evaluate the sentiment in a piece of text as a real number between -1 and 1, a one to five-star rating (e.g. movie or product reviews), or another numeric score outside of any pre-determined interval.\newline
Besides the type of the employed sentiment measurement, the literature can also be categorized according to the unit of analysis it is concerned with. \shortciteA{liu2012book} distinguishes the units of analysis as being either document-level, sentence-level, or entity- and aspect-based. Document-level SA operates separately on every document in a corpus, e.g. each review for a product or movie or each social media post in a collection. This granularity is used most often as it aligns with the nature of a document as a self-contained piece of text by one author. For longer documents, sentence-level SA enables researchers to score sentiment on a per-sentence basis and subsequently calculate aggregate scores for the document. However, this comes at the expense of ignoring the inter-sentence context. Finally, aspect-based SA does not only generate sentiment scores but links these scores to the aspects they are referring to. This is mostly used in the analysis of product reviews, where some aspects of a product are rated positively while others are not, e.g. a computer with great battery life but low compute performance \shortcite{pontiki2016semeval}.

In addition to these definitions which are applicable in the field of NLP, other domains have their own definitions of sentiment. For the domain of finance, \citeA{aggarwal2019defining} provides an overview of all sentiment measures that have been historically used. In recent literature, two of them stick out as the most common: the Chicago Board Options Exchange Volatility Index (VIX) and the Put/Call Ratio (PCR). The calculation of the VIX is based on options on the Standard and Poor's 500 index (S\&P500) and represents the \emph{expected} level of volatility in the next month \shortcite{cboeVIX}. Its forward-looking nature makes it different from usual sentiment measurements which are based on historical data. Given that volatility is defined as the standard deviation of returns, it is not directly comparable to sentiment classes like ``positive'' or ``negative'' although lower volatility is generally associated with higher returns \shortcite{zare2013monetary}. The mathematically simpler Put/Call Ratio is the ratio between the volume of traded put options and call options. A PCR above 1 implies that put options are being traded more than call options indicating a negative market sentiment. Apart from these easily quantifiable versions of market sentiment in finance, the field starts recognizing that retail investors' emotions, sentiments, and opinions carry valuable information. The advent of behavioral finance describes and accounts for human biases in decision making processes, some of which can be assessed using sentiment analysis \shortcite{hirshleifer2015behavioral}.


% -----------------------------------------------------------------------
\subsubsection{Applications of Sentiment Analysis}

Sentiment analysis has a plethora of applications within academia as well as industry. Researchers have used social sentiment to study reactions to adverse events like hurricanes \shortcite{yao2020domain} or the COVID-19 pandemic \shortcite{dubey2020twitter}. Literature regarding political opinion mining demonstrates that sentiment extracted from posts on the microblogging platform Twitter correlates with public opinion to a point where it can be used to forecast election results \shortcite{o2010tweets, tumasjan2010predicting}.
Within finance, social sentiment obtained from microblogging platforms can help forecast stock market volatility \shortcite{antweiler2004all, audrino2020impact}, trading volume \shortcite{oliveira2017impact} and sometimes even future returns \shortcite{ren2018forecasting, wilksch2022predictive}.
In industry, sentiment analysis can be utilized to replace more costly surveys about consumer sentiment or mine information from product reviews. This presents businesses with the opportunity to assess their customers' needs and pain points in a faster, more efficient way and might -- in some cases -- alleviate the need for expensive focus groups. For example, this can be exploited by the film industry which can use social sentiment to forecast box office revenue and movie sales \shortcite{du2014box, rui2013whose}, or manufacturers who survey customers on what they like and dislike about a product \shortcite{pontiki2016semeval}.
In finance, service providers like brokerages are picking up this trend and start offering social sentiment scores as additional investment research to their customers \cite{ibkr-sentiment}. The common denominator between all these use cases is that they profit from more accurate SA models. Albeit automated SA models never perform with flawless accuracy, the closer they are to what humans consider to be ground truth, the better their downstream applications perform: descriptive statistics paint a more faithful picture of reality, forecasting models based on SA yield more accurate predictions (\textcolor{red}{cite!? W/A}), and for researchers, the quality of the obtained data more closely resembles the one of focus group or survey data.


% =======================================================================
\subsection{Automated Sentiment Analysis}

% -----------------------------------------------------------------------
\subsubsection{Data Sets}

Every automated SA model -- whether dictionary-based or machine-learning-based -- requires a set of data to be built upon. For dictionaries, humans analyze large quantities of data to develop sets of words or phrases that carry sentiment and compile them in a machine-usable form. Machine learning models try to automate this process but still need a training data set that is often even larger than the ones used in manual dictionary creation. However, only a limited number of standardized, annotated data sets exist, as it is costly and labor-intensive to create such a data set. Therefore, many of the SA models in the literature have been trained or evaluated on similar data sets. Table \ref{most-used-datasets} provides an overview of the most frequently used data sets for conducting SA on texts from the domain of finance or social media.

\begin{table}[!ht]
	\centering
	\begin{tabular}{llll}
		\toprule
		\textbf{Name} & \textbf{Authors} & \textbf{Source} & \textbf{Size} \\
		\midrule
		Reuters TRC2$^*$ & \shortciteA{reuters-trc2} & news articles & 1.8M \\
		Financial Phrasebank & \shortciteA{malo2014good} & news headlines & 4,837\\
		SemEval-2017 Task 4 & \shortciteA{rosenthal2017semeval} & Twitter & 62,617 \\
		SemEval-2017 Task 5 & \shortciteA{cortis2017semeval} & Twitter \& StockTwits & 2,510 \\
		Fin-SoMe & \shortciteA{chen2020finsome} & StockTwits & 10,000\\
		\bottomrule
		\multicolumn{4}{l}{\footnotesize$^*$ not labeled}
	\end{tabular}
	\caption{Most used data sets for SA in the domain of finance or social media}
	\label{most-used-datasets}
\end{table}

The largest corpus, \emph{Reuters TRC2}, released by \shortciteA{reuters-trc2}, contains 1.8 million Reuters news articles on various topics, including financial markets. However, the data set does not come with any kind of labels and is not exclusively geared towards performing SA. Consequently, it cannot be used for constructing dictionaries or training supervised machine learning models. Despite that, it can be used for pre-training large language models (LLM) which require vast unlabeled corpora for unsupervised training (see section \textcolor{red}{REF LLM sec}). \shortciteA{malo2014good} compiled the \emph{Financial Phrasebank}, a data set of several thousand news headlines where each headline has been annotated as either positive, negative, or neutral by 16 independent annotators. For SA on social media posts, \shortciteA{rosenthal2017semeval} provide \emph{SemEval-2017 Task 4}, a large sample of generic tweets on current events, again labeled as one of three sentiment polarity classes. On the other hand, \emph{SemEval-2017 Task 5} \shortcite{cortis2017semeval} provides data sampled from the financial social media context. The data set contains a subtask (``subtask 1'') which consists of 2,510 labeled messages from StockTwits and Twitter. For each message, three annotators assign each company that is mentioned a sentiment score between -1 and 1. The scores are then consolidated by a fourth expert. The \emph{Fin-SoMe} data set published by \shortciteA{chen2020finsome} consists of 10,000 social media posts from StockTwits. The authors manually annotated the posts, although posts on StockTwits can be assigned a ``bullish'' or ``bearish'' label by the post creator. However, as \shortciteA{chen2020finsome} find, these labels are often wrong.

 It is important to acknowledge the domain a training data set is sampled from as it significantly impacts the performance of the resulting classifier on its target data. For example, \shortciteA{al2020evaluating} show that researchers should always use models that can cope with slang, emojis, and typos when conducting SA on social media data. On the other hand, working with corporate financial filings requires a whole new approach and renders generic models ineffective \shortcite{loughranMcD2011}. Besides the vastly different vocabulary used in different domains, the complexity of a document can also require a change of the unit of analysis. Posts on the social network \emph{Reddit} tend to be much longer than Tweets, which are restricted to 280 characters in the first place. Thus, framing the SA of social media posts as a three-class classification problem might be adequate for Tweets (as a Tweet is likely to fall in one of the positive, negative, or neutral classes), but not so much for Reddit posts. Elaborate texts that span multiple paragraphs and might be a reply to a previous conversation can rarely be categorized as belonging to one of three classes. In these cases, sentence- or paragraph-based analysis is more suitable. \textcolor{blue}{TODO: actually pull Reddit posts and provide descriptive stats?}
 




% -----------------------------------------------------------------------
\subsubsection{Dictionary-based Models}
Automated SA models can be categorized into dictionary- and machine-learning-based methods. Dictionary-based SA is a common approach that is liked for its simplicity. SA dictionaries are word lists that assign each word a score. Simple scoring methods might classify single words as positive or negative, for example LIWC \shortcite{pennebaker2001linguistic}, Harvard General Inquirer \shortcite{stone1966general}, or Opinion Observer \shortcite{liu2005opinion}. Others rate them on more sophisticated numeric scales, like ANEW \shortcite{bradley1999affective}, SentiWordNet \shortcite{baccianella2010sentiwordnet}, or VADER \shortcite{hutto2014vader}. The sentiment for a document is subsequently calculated as an aggregate of all word scores. This methodology makes dictionaries explainable and computationally cheap at inference time but does not come without drawbacks. First, the dictionary needs to be compiled by humans which is time-consuming and requires decisions regarding scales and scoring which significantly impact the performance of the final model. Second, the rigid approach of gathering a list of words can fail if documents contain few or none of the words in the list. This makes it especially hard to apply lexicon-based approaches on social media content which is riddled with typos, slang words, and emojis. Models that are not designed for this type of content often classify texts as neutral, simply for the lack of any matching words. Finally, the typical challenges of SA (see section \ref{section-sa-challenges}) have to be addressed manually. For example, \citeA{hutto2014vader} designed VADER by integrating a set of heuristics for handling negation, punctuation, and capitalization as degree modifiers of sentiment. These features make VADER particularly attractive to researchers working with social media content, as the integrated heuristics make it outperform all its competitors on social media content \cite{al2020evaluating}.

>> more?
% -----------------------------------------------------------------------
\subsubsection{Machine-Learning-based Models}
\begin{itemize}[noitemsep]
	\item Note: there aren't really any non-DL ML models out there. Tho: there is a lot of research on ML models, buuut there are no published artifacts. These are mostly dictionaries/DL!
	\item Recurrent models
	\item Transformer models
	\item LLMs
\end{itemize}


\emph{Argument:} We can only consider models that are available as artifacts. Theoretical papers cannot be considered, as we can't reproduce their models/apply them in our experiments




\subsubsection{Challenges of Sentiment Analysis}
\label{section-sa-challenges}
Considering the ambiguous nature of sentiments and opinions, analyzing them entails multiple challenges that need to be dealt with. According to \shortciteA{hussein2018survey}, the most common SA challenges are negation handling, domain dependence, spam detection, and ambiguity in the form of abbreviations or sarcasm. Negation handling presents an issue because very few words that might not be in direct proximity to the sentiment-laden part of a sentence can completely invert its meaning. In combination with domain-specific vocabulary, this can even be hard to spot for human annotators. For example, in finance, ``buying calls'' indicates a bullish sentiment whereas ``buying puts'' conveys a bearish sentiment toward a given stock. However, the nature of stock options as financial instruments allows a market participant to also \emph{sell} options, in which case the sentiment is reverted, and ``selling calls'' and ``selling puts'' convey bearish and bullish sentiments respectively. Thus, neither the words ``buy'' and ``sell'', nor the words ``put'' and ``call'' can be assigned a clearly positive or negative sentiment. This demonstrates the importance of context within a document and is particularly challenging for dictionary-based models which score documents on a word-by-word basis. Moreover, domain-specificity is not only a problem when it occurs in conjunction with negation. Different vocabulary, idioms, slang, and divergent interpretations of common words between domains can significantly degrade the quality of a SA. \citeA{ravi2015survey} provide an overview of work that addresses the challenge of cross-domain SA, but conclude that it is still an unsolved problem. Consequently, researchers should be careful when applying generic SA techniques to domain-specific corpora and vice versa. On data sets obtained from social media platforms, spam detection presents another issue that needs careful consideration. Many posts on social media are advertisements or were created by automated robots that post similar content multiple times. Not only can such duplicates ruin the quality of a collected data set, but they also dilute the content posted by real humans as spammers try to blend in as much as possible. Removing spam is viable through heuristics developed after manual inspection of a data set, for example by using word lists \shortcite{yao2020domain}. However, the precision of such methods must be scrutinized to not remove too much informative human-created content. Arguably the hardest challenge is coping with ambiguity and sarcasm. Using text as a medium of exchange of opinions can make these stylistic devices hard to identify even for humans. The sentence ``Yeah, company X is the best investment in the world...'' requires intonation or other cues to convey whether this is a sarcastic note or a serious opinion. This makes SA a problem on which even humans might not unanimously agree. If such uncertainty is present in a corpus of labeled data, it will also be present in any SA model that is developed based on this corpus.







% =======================================================================
\subsection{Research Gap}
\begin{itemize}[noitemsep]
	\item There exist data sets and models, mostly for StockTwits\dots
	\begin{itemize}[noitemsep]
		\item Data sets: SemEval Task 5 (StockTwits + Twitter), Fin-SoMe (StockTwits)
		\item Models: Pontes21, NTUSD-Fin (Chen et al., 2018), Li17
	\end{itemize}
	\item No data set on only Twitter
\end{itemize}
-------------------------------------------------
\begin{itemize}[noitemsep]
	\item performance benchmarks of LLMs (and all other models too, right?) on domain-specific texts
	\item We develop a data set that is a) finance-specific and b) sampled from a social media context (Twitter!) 	
\end{itemize}

\begin{table}[!ht]
\centering
\begin{tabular}{ccc}
	\toprule
	& \multicolumn{2}{c}{\textbf{Finance-Specific}} \\
	\cmidrule(l){2-3}
	\textbf{SNS-Specific} & Yes & No\\
	\midrule
	Yes & \makecell{\shortciteA{sohangir2018bigdata}$^{*\dagger}$ \\ NTUSD-Fin \shortcite{chen2018ntusd}} & \makecell{
		SentiStrength \shortcite{sentistrength},\\
		AFINN \shortcite{nielsen2011new},\\
		VADER \shortcite{hutto2014vader},\\
		Twitter roBERTa \shortcite{barbieri2020tweeteval}$^\dagger$
		}\\
	\midrule
	No & \makecell{
		\shortciteA{loughranMcD2011},\\
		FinBERT \shortcite{araci2019finbert}$^\dagger$
		} & \makecell{
		Harvard-IV-4 \shortcite{stone1966general},\\
		ANEW \shortcite{bradley1999affective},\\
		LIWC \shortcite{pennebaker2001linguistic},\\
		Opinion Observer \shortcite{liu2005opinion},\\
		SentiWordNet \shortcite{baccianella2010sentiwordnet}
		} \\
	\bottomrule
	\multicolumn{3}{l}{\makecell[l]{\footnotesize{* model artifact has not been published}\\ \footnotesize{$\dagger$ deep-learning-based model}}}
\end{tabular}
\caption{Overview of sentiment analysis models by domain}
\label{table-research-gap}
\end{table}





% =======================================================================
\subsection{Conceptual Framework}

In this work, \dots 

\begin{itemize}[noitemsep]
	\item In this work: focus on SA and opinion mining (i.e. everything on uni-dimensional scales!) as defined in the field of NLP. This ``Social Sentiment'' definition is also slowly getting traction in finance, btw. Also here in this work: document-level!
\end{itemize}



\begin{figure}[!ht]
	\centering
	\begin{tikzpicture}

	% --- NODES ---
	\node(m) at (-3, 0) {
		\begin{myboxprimary}{Model}{width=4.5cm}
		\begin{itemize}[noitemsep,leftmargin=*]
			\item Dictionary
			\item Machine Learning
			\item Deep Learning
		\end{itemize}	
		\end{myboxprimary}
	};
	
	\node(d) at (5, 0) {
		\begin{myboxprimary}{Dataset}{width=5cm}
		\begin{itemize}[noitemsep,leftmargin=*]
			\item SemEval2017
			\item FinancialPhrasebank
			\item Reuters TRC2
		\end{itemize}	
		\end{myboxprimary}
	};
	
	
	\node(t) at (2.5, -3.75) {
		\begin{myboxsecondary}{Task}{width=4.1cm}
		\begin{itemize}[noitemsep,leftmargin=*]
			\item 3-class polarity
			\item Continuous scale
			\item Ordinal scale
		\end{itemize}	
		\end{myboxsecondary}
	};
	
	\node(domain) at (7.5, -3.75) {
		\begin{myboxsecondary}{Domain}{width=4cm}
		\begin{itemize}[noitemsep,leftmargin=*]
			\item Twitter
			\item StockTwits
			\item News Headlines
		\end{itemize}	
		\end{myboxsecondary}
	};
	
	
	\node[draw,fit=(m)(d)(t)(domain),dashed,label=\textbf{Sentiment Analysis}]{};

	% --- ARROWS ---	
	\draw[->,line width=1pt,] (m) -- (d) node[midway,above]{based on};
	\draw[<-,line width=1pt] (t.north) -- +(0,0.4) -| (d.240) node[midway,left,yshift=0.3cm]{defines};
	
	\draw[<-,line width=1pt] (domain.north) -- +(0,0.4) -| (d.300) node[midway,right,yshift=0.3cm]{sampled from};
	
	\draw[<-,line width=1pt] (t.west) -| (m.south) node[midway,right,yshift=0.3cm]{performs};
	
	\end{tikzpicture}
\end{figure}

\emph{Research Questions}: Do LLMs perform well on domain-specific sentiment ananylsis tasks? How do they compare to smaller, machine-learning-based models as well as generic off-the-shelf models? Can simple, domain-specific models outperform LLMs/fine-tuned LLMs? Even if not, how do their memory/compute/cost footprints compare? 

Models to build:

\begin{table}[!ht]
\centering
	\begin{tabular}{ccc}
		\toprule
%		\textbf{Machine Learning} & \textbf{Deep Learning} & \textbf{Transfer Learning}\\
%		\midrule
%		Naive Bayes & Recurrent NN & BERT \\
%		Logistic Regression & Convolutional NN & DistillBERT \\
%		Random Forest & & roBERTa \\
%		LightGBM & & \\

%		\midrule
		\multicolumn{2}{c}{\textbf{Trained from Scratch}} & \textbf{Transfer Learning}\\
		\cmidrule(r){1-2} \cmidrule(l){3-3}
		\emph{Machine Learning} & \emph{Deep Learning} & \emph{Large Language Models}\\
		\midrule
		Naive Bayes & Recurrent NN & BERT \\
		Logistic Regression & Convolutional NN & DistillBERT \\
		Random Forest? & & roBERTa \\
		LightGBM? & & \\
		\bottomrule		
	\end{tabular}
	\caption{Models to build}
\end{table}








