\section{Theoretical Background}

\subsection{Financial Posts on Social Media}
\begin{itemize}[noitemsep]
	\item Platforms: Twitter, StockTwits, Reddit, Crypto-specific stuff
	\item Data characteristica: Reddit posts vs. Twitter posts (compare avg. length of each?)
\end{itemize}


\subsection{Sentiment Analysis}  % TODO: wording
\subsubsection{Operationalization of Sentiment}
\begin{itemize}[noitemsep]
	\item Definition: ``Opinion'' would be a more accurate term (Munezero, 2014)
	\item Scales: continuous $\in [-1, 1]$, discrete $\in \{pos, neg, neu\}$, 1-5 star reviews \dots
\end{itemize}
\subsubsection{Sentiment in the Financial Context}

Finance-specific, alternative operationalization of sentiment:
\begin{itemize}[noitemsep]
	\item VIX
	\item Fear \& greed index, more in Aggarwal (2018)
	\item Social sentiment @IBKR?
	
\end{itemize}

\subsection{Automated Sentiment}

\begin{figure}
	\centering
	\includegraphics[width=0.8\textwidth]{assets/images/plot.pdf}
	\caption{This is the caption, 1234567890}
\end{figure}

\subsubsection{Model Perspective}
\subsubsection{Data Set Perspective}
Mention data-centric AI!
\begin{itemize}[noitemsep]
	\item describe different task types
	\item describe different domains + challenges
\end{itemize}



An example of a prompt to GPT-J \cite{gpt-j}



\subsection{Conceptual Framework}

\begin{figure}[!ht]
	\centering
	\begin{tikzpicture}

	% --- NODES ---
	\node(m) at (-3, 0) {
		\begin{myboxprimary}{Model}{width=4.5cm}
		\begin{itemize}[noitemsep,leftmargin=*]
			\item Dictionary
			\item Machine Learning
			\item Deep Learning
		\end{itemize}	
		\end{myboxprimary}
	};
	
	\node(d) at (5, 0) {
		\begin{myboxprimary}{Dataset}{width=5cm}
		\begin{itemize}[noitemsep,leftmargin=*]
			\item SemEval2017
			\item FinancialPhrasebank
			\item Reuters TRC2
		\end{itemize}	
		\end{myboxprimary}
	};
	
	
	\node(t) at (2.5, -3.75) {
		\begin{myboxsecondary}{Task}{width=4.1cm}
		\begin{itemize}[noitemsep,leftmargin=*]
			\item 3-class polarity
			\item Continuous scale
			\item Ordinal scale
		\end{itemize}	
		\end{myboxsecondary}
	};
	
	\node(domain) at (7.5, -3.75) {
		\begin{myboxsecondary}{Domain}{width=4cm}
		\begin{itemize}[noitemsep,leftmargin=*]
			\item Twitter
			\item StockTwits
			\item News Headlines
		\end{itemize}	
		\end{myboxsecondary}
	};
	
	
	\node[draw,fit=(m)(d)(t)(domain),dashed,label=\textbf{Sentiment Analysis}]{};

	% --- ARROWS ---	
	\draw[->,line width=1pt,] (m) -- (d) node[midway,above]{based on};
	\draw[<-,line width=1pt] (t.north) -- +(0,0.4) -| (d.240) node[midway,left,yshift=0.3cm]{defines};
	
	\draw[<-,line width=1pt] (domain.north) -- +(0,0.4) -| (d.300) node[midway,right,yshift=0.3cm]{sampled from};
	
	\draw[<-,line width=1pt] (t.west) -| (m.south) node[midway,right,yshift=0.3cm]{performs};
	
	\end{tikzpicture}
\end{figure}