\section{Theoretical Background}



% =======================================================================
\subsection{Sentiment Analysis}  % TODO: wording

% -----------------------------------------------------------------------
\subsubsection{Operationalization of Sentiment}
\begin{itemize}[noitemsep]
	\item Definition: ``Opinion'' would be a more accurate term (Munezero, 2014)
	\item Scales: continuous $\in [-1, 1]$, discrete $\in \{pos, neg, neu\}$, 1-5 star reviews \dots
\end{itemize}

% -----------------------------------------------------------------------
\subsubsection{Sentiment in the Financial Context}

Finance-specific, alternative operationalization of sentiment:
\begin{itemize}[noitemsep]
	\item VIX
	\item Fear \& greed index, more in Aggarwal (2018)
	\item Social sentiment @IBKR?
	
\end{itemize}

% =======================================================================
\subsection{Automated Sentiment Analysis}


% -----------------------------------------------------------------------
\subsubsection{Model Perspective}
\begin{itemize}[noitemsep]
	\item Dictionaries
	\item ML
	\item DL (RNN, Transformer models)
\end{itemize}

\emph{Argument:} We can only consider models that are available as artifacts. Theoretical papers cannot be considered, as we can't reproduce their models/apply them in our experiments




% -----------------------------------------------------------------------
\subsubsection{Data Set Perspective}
Mention data-centric AI!
\begin{itemize}[noitemsep]
	\item describe different task types
	\item describe different domains + challenges (financial posts on social media)
	\begin{itemize}[noitemsep]
		\item Platforms: Twitter, StockTwits, Reddit, Crypto-specific stuff
		\item Data characteristica: Reddit posts vs. Twitter posts (compare avg. length of each?)
\end{itemize}
\end{itemize}



An example of a prompt to GPT-J \cite{gpt-j}

% =======================================================================
\subsection{Large Language Models}

% =======================================================================
\subsection{Research Gap?}
\begin{itemize}[noitemsep]
	\item performance benchmarks of LLMs (and all other models too, right?) on domain-specific texts
	\item development of a data set 	
\end{itemize}

\begin{table}[!ht]
\centering
\begin{tabular}{ccc}
	\toprule
	& \multicolumn{2}{c}{\textbf{Finance-Specific}} \\
	\cmidrule(l){2-3}
	\textbf{SNS-Specific} & Yes & No\\
	\midrule
	Yes & Sohangir (2018)$^*$ & \makecell{VADER, AFINN,\\ SentiStrength, Twitter roBERTa} \\[15pt]
	No & \makecell{Loughran\&McDonald,\\ FinBERT} & \makecell{SentiWordNet, ANEW,\\ LIWC, Harvard-IV-4, Hu \& Liu}\\
	\bottomrule
	\multicolumn{3}{l}{\footnotesize{* model artifact has not been published}}
\end{tabular}
\end{table}





% =======================================================================
\subsection{Conceptual Framework}

\begin{figure}[!ht]
	\centering
	\begin{tikzpicture}

	% --- NODES ---
	\node(m) at (-3, 0) {
		\begin{myboxprimary}{Model}{width=4.5cm}
		\begin{itemize}[noitemsep,leftmargin=*]
			\item Dictionary
			\item Machine Learning
			\item Deep Learning
		\end{itemize}	
		\end{myboxprimary}
	};
	
	\node(d) at (5, 0) {
		\begin{myboxprimary}{Dataset}{width=5cm}
		\begin{itemize}[noitemsep,leftmargin=*]
			\item SemEval2017
			\item FinancialPhrasebank
			\item Reuters TRC2
		\end{itemize}	
		\end{myboxprimary}
	};
	
	
	\node(t) at (2.5, -3.75) {
		\begin{myboxsecondary}{Task}{width=4.1cm}
		\begin{itemize}[noitemsep,leftmargin=*]
			\item 3-class polarity
			\item Continuous scale
			\item Ordinal scale
		\end{itemize}	
		\end{myboxsecondary}
	};
	
	\node(domain) at (7.5, -3.75) {
		\begin{myboxsecondary}{Domain}{width=4cm}
		\begin{itemize}[noitemsep,leftmargin=*]
			\item Twitter
			\item StockTwits
			\item News Headlines
		\end{itemize}	
		\end{myboxsecondary}
	};
	
	
	\node[draw,fit=(m)(d)(t)(domain),dashed,label=\textbf{Sentiment Analysis}]{};

	% --- ARROWS ---	
	\draw[->,line width=1pt,] (m) -- (d) node[midway,above]{based on};
	\draw[<-,line width=1pt] (t.north) -- +(0,0.4) -| (d.240) node[midway,left,yshift=0.3cm]{defines};
	
	\draw[<-,line width=1pt] (domain.north) -- +(0,0.4) -| (d.300) node[midway,right,yshift=0.3cm]{sampled from};
	
	\draw[<-,line width=1pt] (t.west) -| (m.south) node[midway,right,yshift=0.3cm]{performs};
	
	\end{tikzpicture}
\end{figure}

\emph{Research Questions}: Do LLMs perform well on domain-specific sentiment ananylsis tasks? How do they compare to smaller, machine-learning-based models as well as generic off-the-shelf models? Can simple, domain-specific models outperform LLMs/fine-tuned LLMs? Even if not, how do their memory/compute/cost footprints compare? 

Models to build:

\begin{table}[!ht]
\centering
	\begin{tabular}{ccc}
		\toprule
%		\textbf{Machine Learning} & \textbf{Deep Learning} & \textbf{Transfer Learning}\\
%		\midrule
%		Naive Bayes & Recurrent NN & BERT \\
%		Logistic Regression & Convolutional NN & DistillBERT \\
%		Random Forest & & roBERTa \\
%		LightGBM & & \\

%		\midrule
		\multicolumn{2}{c}{\textbf{Trained from Scratch}} & \textbf{Transfer Learning}\\
		\cmidrule(r){1-2} \cmidrule(l){3-3}
		\emph{Machine Learning} & \emph{Deep Learning} & \emph{Large Language Models}\\
		\midrule
		Naive Bayes & Recurrent NN & BERT \\
		Logistic Regression & Convolutional NN & DistillBERT \\
		Random Forest? & & roBERTa \\
		LightGBM? & & \\
		\bottomrule		
	\end{tabular}
	\caption{Models to build}
\end{table}








