 \renewcommand{\abstractname}{Zusammenfassung}
\begin{abstract}
\noindent

% Die Qualität der Vorhersagen von generischen automatisierten Sentiment-Analyse-Modellen auf domäne-spezifischem Text ist durchwachsen. Um die Ausmaße dieses Problems zu quantifizieren und zu seiner Lösung beizutragen, sammelt diese Arbeit einen Datensatz aus 10,000 Tweets, die die Themen Finnazen und Investieren diskutieren. Den tweets wird manuell ihr Marktsentiment zugewiesen, also die Annahmen der Autoren bezüglich zukünftiger Entwicklung von Aktienkursen. Basierend auf diesem Datensatz zeigen wir, dass alle existieren Sentiment-Analyse-Modelle, die auf angrenzenden Themengebieten trainiert wurden, Probleme haben, akkurate Ergebnisse zu liefern.


\noindent Als Reaktion auf die schwache Leistung generischer automatischer Sentiment-Analyse-Modelle für domänenspezifische Texte sammeln wir einen Datensatz von 10.000 Twitter-Beiträgen, die sich mit den Themen Finanzen und Investitionen befassen. Wir ordnen den Tweets manuell ihr Markt-Sentiment zu, also die Erwartung der Autoren hinsichtlich der zukünftigen Entwicklung einer Aktie. Basierend auf diesem Datensatz zeigen wir, dass alle bestehenden Sentiment-Modelle, die auf ähnlichen Domänen trainiert wurden, aufgrund des fachspezifischen Vokabulars dieser Aufgabe nicht in der Lage sind, eine akkurate Sentiment-Analyse durchzuführen. Daher entwickeln, trainieren und implementieren wir unser eigenes Sentiment-Modell. Es übertrifft alle bisherigen Modelle bei der Klassifizierung des Markt-Sentiments in Tweets und schneidet selbst bei der Anwendung auf Posts einer anderen Plattform gleich gut ab wie BERT-basierte large language models. Aufgrund des einfachen Designs des Modells erreichen wir dieses Ergebnis zu einem Bruchteil der Trainings- und Inferenzkosten. Wir veröffentlichen das trainierte Modell-Artefakt als Python-Bibliothek, um seine Verwendung in zukünftiger Forschung und Praxis zu erleichtern.\newline

\providecommand{\keywords}[1]
{
	\small
 	\noindent\textbf{Schlüsselwörter: } #1
}
\keywords{Sentiment-Analyse, Finanzmarkt-Sentiment, Opinion Mining, Maschinelles Lernen, Deep Learning}

\end{abstract}