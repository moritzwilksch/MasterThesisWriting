\section{Methodology}
% =======================================================================
\subsection{Data Collection}
% -----------------------------------------------------------------------
\subsubsection{Data Sources}
English-speaking users who discuss finance and investing on online social media platforms in text form do so on three major platforms: Reddit, StockTwits, and Twitter. Reddit is a SNS on which users can create their own communities (``subreddits'') that focus on a specific topic. For example, users have created the subreddits ``Investing'' and ``StockMarket'' to discuss long-term investments and the subreddit ``WallStreetBets'' for posts about high-risk short-term gambles in the market. However, posts on Reddit tend to be much longer than posts on Twitter or StockTwits. Their length would require them to be analyzed on a paragraph- or sentence basis. Since research on SA is mostly focused on document-level analysis, we will not use Reddit posts for this work. The decision between StockTwits and Twitter is harder: posts on both platforms are similar in length and share the usage of cashtags (a ``\$'' sign followed by a ticker symbol) for identifying stocks. We decide to obtain data from Twitter rather than StockTwits for the following reasons:
\begin{enumerate}[noitemsep]
	\item The post volume on Twitter is higher than it is on StockTwits.
	\item The few data sets for SA on financial SM posts that exist use data from StockTwits, hence publishing a data set of Tweets provides more value to the research community.
	\item By using Twitter data for our experiments we can answer RQ4 and compare performances between models and data sources.
\end{enumerate}

A disadvantage of the Twitter platform is that -- unlike StockTwits -- the majority of tweets are not related to finance and investing. As a remedy, we utilize cashtags for searching investment-related posts on Twitter. These tags are only used when referring to publicly traded companies as financial entities, as each cashtag contains the company's stock ticker symbol. This mostly prevents generic tweets about a company's brand or products from spilling into the collected data and allows us to focus the artifact design process on financial social media posts.

% -----------------------------------------------------------------------
\subsubsection{Sampling}

The first step to collecting data on Twitter is assembling a search query because the Twitter search application programming interface (API) requires users to search for specific cashtags instead of any Tweet containing cashtags. To make results comparable to the previous literature we will focus on English posts only. Therefore, as a starting point for selecting ticker symbols to include in the search query, we use the S\&P500 index. From there, we impose a minimum activity filter on each stock ticker: a ticker is only considered to be actively discussed on Twitter if there are more than 100 tweets per day on average mentioning it. Without such a filter, there is a risk of few Twitter users representing the ``public'' sentiment of a company. To conduct the filtering, we collect data on the number of tweets per day for every S\&P500 ticker during April of 2022. As Figure \ref{figure-tweet-activity} demonstrates, the distribution of activity is highly skewed. The top 20 tickers account for 53.7\% of all tweets about S\&P500 companies. 56 tickers fulfill the minimum activity constraint and account for 70.9\% of tweet volume. Out of these 56, we manually exclude 6 tickers (\texttt{AME, OGN, TEL, AMP, KEY, STX}) because while they represent corporations listed in the S\&P500 index, they are mostly used to reference cryptocurrencies on Twitter. This is problematic as the domain of cryptocurrencies is fundamentally different from the equity market. Financial instruments like options do not exist for cryptocurrencies and they are also not affected by any kind of fundamental information. Thus, we decide to remove them from the data set and focus on publicly traded companies.

\begin{figure}
	\includegraphics[width=\textwidth]{assets/images/tweet_counts.pdf}
	\caption{Number of average Tweets per day in April 2022 for the top 20 tickers}
	\label{figure-tweet-activity}
\end{figure}

The top 20 tickers (out of 505) produced 53.7\% of all tweets, top 56 produced 70.9\%
\begin{itemize}[noitemsep]
	\item Twitter: convenience sampling of last $7d$
\end{itemize}


% =======================================================================
\subsection{Data Labelling}

% -----------------------------------------------------------------------
\subsubsection{Task Definition}
\begin{itemize}[noitemsep]
	\item Difference Market Sentiment vs. Writer Sentiment (see FinSoMe, chen et al):
\end{itemize}


\begin{table}[!ht]
\centering
\small
\begin{tabular}{p{7.5cm}p{7.5cm}}
\toprule
\multicolumn{1}{c}{\textbf{Positive}} & \multicolumn{1}{c}{\textbf{Negative}} \\
\midrule

\begin{itemize}[noitemsep,leftmargin=*,topsep=-12pt]
	\item bought stock, holding stock, \emph{not} selling stock, want to buy stock (positive expectations)
	\item buying calls, selling puts, being long
	\item stock is a bargain, undervalued, oversold, reaching all-time high, is in an up trend
	\item positive earnings release, growing revenue, profits, or customer base, \emph{not} absolute numbers without judgement or direction
	\item price target raised
	\item made a profit, praising or giving a positive example, asking positive rhetorical question
	\item business acquisitions \& expansions

\end{itemize} & \begin{itemize}[noitemsep,leftmargin=*,topsep=-12pt]
	\item holding or selling with disappointing return
	\item buying puts, selling calls, being short
	\item not buying or selling for negative expectations
 	\item stock is overvalued, overbought, reaching a new low, is in a down trend
 	\item negative news like law suits or bad press
 	\item lowered price target
 	\item \textcolor{blue}{gloating, enjoying other's misfortune}
 	\item insulting, mocking, or giving a negative example
 	\item rhetorical questions suggesting negative sentiment 

\end{itemize}\\
\toprule
\multicolumn{1}{c}{\textbf{Uncertain}} & \multicolumn{1}{c}{\textbf{No Sentiment}} \\
\midrule

\begin{itemize}[noitemsep,leftmargin=*,topsep=-12pt]
	\item buying, selling, or not investing for uncertain expectations
	\item list pro \emph{and} con arguments for investment
	\item list positive \emph{and} negative opinions or facts in same Tweet
	\item asking for guidance because of uncertainty
	\item \emph{not:} changing one's mind, the more recent opinion counts

\end{itemize} & \begin{itemize}[noitemsep,leftmargin=*,topsep=-12pt]
	\item neutral information or news headlines
	\item absolute numbers without directional interpretation, stating non-opinionated facts that are not inherently positive or negative
	\item spam, ads, or not related to topic of investing
	\item seeking other's opinions
	\item stating an opinion which does not contain a positive or negative sentiment
\end{itemize}\\

\bottomrule
\end{tabular}
\caption{The codebook which guides the data labeling}
\label{table-codebook}
\end{table}


% -----------------------------------------------------------------------
\subsubsection{Data Quality Assessment}  % TODO: wording
\begin{itemize}[noitemsep]
	\item de-dupe
	\item length/cashtag filter
	\item potentially: spam classifier model?
	\item mention: no active learning (\url{https://blog.fastforwardlabs.com/2019/04/02/a-guide-to-learning-with-limited-labeled-data.html})
\end{itemize}




% =======================================================================
\subsection{Data Preprocessing}
\label{section-data-preprocessing}
% -----------------------------------------------------------------------
\subsubsection{Tokenization}
\begin{itemize}[noitemsep]
	\item tokenization
\end{itemize}

% -----------------------------------------------------------------------
\subsubsection{Representation of Text Data}
\begin{itemize}[noitemsep]
	\item vectorizer (TF-IDF?)
	\item Embeddings?
\end{itemize}






% =======================================================================
\subsection{Experiment Design}

%todo: this needs a section for descriptive experiments like bias check and sample tweets demo

% -----------------------------------------------------------------------
\subsubsection{Performance Evaluation}
\begin{itemize}[noitemsep]
	\item cross validation
	\item metrics
\end{itemize}


\begin{figure}[!ht]
	\centering
	\small
	\begin{tikzpicture}

	\matrix[matrix of nodes,nodes={draw,minimum width=1em,anchor=south,align=center,minimum height=1.8em,text width=5em},row sep=0.5em,column sep=-\pgflinewidth](mtx){|[fill=grey]|$t_1$ & $d_{1,1}$ & $d_{1,2}$ & $d_{1,3}$ & $d_{1,4}$ \\
	$d_{2,1}$ & |[fill=grey]|$t_2$ & $d_{2,2}$ & $d_{2,3}$ & $d_{2,4}$\\	
	$d_{3,1}$ & $d_{3,2}$ & |[fill=grey]|$t_3$ & $d_{3,3}$ & $d_{3,4}$\\		
	$d_{4,1}$ & $d_{4,2}$ & $d_{4,3}$ & |[fill=grey]|$t_4$ &  $d_{3,4}$\\		
	$d_{5,1}$ & $d_{5,2}$ & $d_{5,3}$ &  $d_{5, 4}$ & |[fill=grey]|$t_5$ \\		
	};
	
	\node[left=2em of mtx-1-1,anchor=east]{Split 1};
	\node[left=2em of mtx-2-1,anchor=east]{Split 2};	
	\node[left=2em of mtx-3-1,anchor=east]{Split 3};
	\node[left=2em of mtx-4-1,anchor=east]{Split 4};
	\node[left=2em of mtx-5-1,anchor=east]{Split 5};		

	
	\end{tikzpicture}
	\caption{Nested cross-validation split of a dataset. Grey shade represents data used for testing, white represents data used for training and validation.}
	\label{figure-cv-split}
\end{figure}


% -----------------------------------------------------------------------
\subsubsection{Selection of Models}


Models to build:

\begin{table}[!ht]
\centering
	\begin{tabular}{ccc}
		\toprule
%		\textbf{Machine Learning} & \textbf{Deep Learning} & \textbf{Transfer Learning}\\
%		\midrule
%		Naive Bayes & Recurrent NN & BERT \\
%		Logistic Regression & Convolutional NN & DistillBERT \\
%		Random Forest & & roBERTa \\
%		LightGBM & & \\

%		\midrule
		\multicolumn{2}{c}{\textbf{Trained from Scratch}} & \textbf{Transfer Learning}\\
		\cmidrule(r){1-2} \cmidrule(l){3-3}
		\emph{Machine Learning} & \emph{Deep Learning} & \emph{Large Language Models}\\
		\midrule
		Naive Bayes & Recurrent NN & BERT \\
		Logistic Regression & Convolutional NN & DistillBERT \\
		Random Forest? & & roBERTa \\
		LightGBM? & & \\
		\bottomrule		
	\end{tabular}
	\caption{Models to build}
\end{table}



\begin{itemize}[noitemsep]
	\item model types
	\item hyperparameters
\end{itemize}

% -----------------------------------------------------------------------
\subsubsection{Hyperparameters/Search Space?}


\textbf{Finally:} Show a flowchart of everything?






